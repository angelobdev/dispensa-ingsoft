\chapter{Design Pattern}

\section{Introduzione}

Un \textbf{Design Pattern} è una soluzione pronta all’uso o preliminarmente adattabile per specifici problemi applicativi. I Design Pattern affrontano questioni rilevanti come la riusabilità, la modificabilità, l’estendibilità e il disaccoppiamento dei moduli software. La loro conoscenza offre un linguaggio comune (sintassi e semantica) per facilitare la comunicazione all’interno dei team di progetto.

I Design Pattern si classificano in tre categorie principali:
\begin{itemize}
    \item \textbf{Creazionali}: si occupano della creazione degli oggetti.
    \item \textbf{Strutturali}: descrivono come comporre classi e oggetti.
    \item \textbf{Comportamentali}: si concentrano sulle interazioni tra oggetti e algoritmi.
\end{itemize}

Questa classificazione, sebbene ampiamente accettata, non è universalmente condivisa. Ad esempio, il pattern \textit{State} appartiene alla categoria comportamentale. I Design Pattern possono essere rappresentati tramite diagrammi UML che coinvolgono classi, interfacce, oggetti e le loro relazioni.

I \textbf{23 Design Pattern di riferimento} sono descritti nel libro \textit{Design Patterns: Elements of Reusable Object-Oriented Software} (Gang of Four: E. Gamma, R. Helm, R. Johnson, J. Vlissides, 1995). Tuttavia, nuovi pattern continuano a essere scoperti e documentati.

\subsection{Benefici dei Design Pattern}
L’uso dei Design Pattern facilita il riuso efficace di progetti e architetture, rendendo accessibili agli sviluppatori tecniche consolidate. Aiutano a scegliere tra alternative progettuali che migliorano la riusabilità del sistema e a scartare quelle che la comprometterebbero.

I Design Pattern rispondono a esigenze quali:
\begin{itemize}
    \item Isolare e incapsulare gli aspetti che cambiano in un progetto.
    \item Applicare il principio \textbf{Open/Closed}: apertura ai cambiamenti tramite l’aggiunta di nuovi moduli, mantenendo invariati gli altri.
    \item Migliorare la documentazione e la manutenzione dei sistemi esistenti.
\end{itemize}

\subsection{Struttura di un Design Pattern}
Ogni Design Pattern è caratterizzato da quattro elementi essenziali:
\begin{itemize}
    \item \textbf{Nome}: descrive sinteticamente il problema di progettazione, la soluzione e le conseguenze della soluzione scelta.
    \item \textbf{Problema (o intento)}: spiega il contesto e il problema che il pattern risolve.
    \item \textbf{Soluzione (concettuale)}: descrive gli elementi del progetto, le loro relazioni e collaborazioni. Non fornisce un’implementazione, ma una configurazione astratta.
    \item \textbf{Conseguenze}: descrivono i risultati e i vincoli derivanti dall’applicazione del pattern, aiutando a valutare soluzioni alternative e stimare costi e benefici.
\end{itemize}

\subsection{Origini dei Design Pattern}
Il concetto di pattern è stato introdotto da \textbf{Christopher Alexander} nel suo libro \textit{A Pattern Language: Towns, Buildings, Constructions} (Oxford University Press, 1977). Alexander descrisse circa 250 pattern per la progettazione architettonica di case e città, fornendo per ciascuno un nome, un contesto d’uso, il problema da risolvere e la soluzione proposta.

Un esempio di pattern architettonico è il \textit{Corridoio corto}, che suggerisce di evitare corridoi più lunghi di 16-17 metri per ridurre il disagio causato dalla loro oscurità e desolazione. La soluzione consiste nel progettare corridoi corti con ampie vetrate e rientranze arredate.

\subsection{Classificazione dei Design Pattern}
I Design Pattern possono essere classificati secondo due criteri:
\begin{itemize}
    \item \textbf{Scopo}: indica ciò che il pattern fa (Creazionali, Strutturali, Comportamentali).
    \item \textbf{Raggio d’azione}: distingue tra pattern che riguardano le relazioni statiche tra classi (\textit{Class-pattern}) e quelli che riguardano le relazioni dinamiche tra oggetti (\textit{Object-pattern}).
\end{itemize}

La maggior parte dei Design Pattern ha gli oggetti come raggio d’azione, ma alcuni si concentrano sulle classi. Ad esempio:
\begin{itemize}
    \item \textbf{Pattern Creazionali}: delegano parte del processo di creazione di un oggetto alle sottoclassi o ad altri oggetti.
    \item \textbf{Pattern Strutturali}: utilizzano l’ereditarietà per comporre classi o descrivono modi per raggruppare oggetti.
    \item \textbf{Pattern Comportamentali}: descrivono algoritmi e flussi di controllo o il modo in cui gruppi di oggetti cooperano per svolgere attività complesse.
\end{itemize}

\newpage

\section{Creazionali}

I pattern creazionali permettono di astrarre il processo di creazione. 
\begin{itemize}
    \item \textbf{Class Creational}: utilizzano l'ereditarietà per variare le classi istanziate \textit{(inheritance)};
    \item \textbf{Object Creational}: delegano il processo ad un altro oggetto \textit{(delegation or composition);}
\end{itemize}
Inoltre forniscono molta flessibilità su \textit{cosa} viene creato, \textit{chi} crea, \textit{come} viene creato e \textit{quando}.

\subsection{Abstract Factory}
\label{abstract-factory}

\textbf{Scopo}: Creazionale \\
\textbf{Raggio d'azione}: Oggetti

\paragraph{Definizione} Fornisce un’interfaccia per la creazione di famiglie di oggetti correlati o dipendenti senza specificare quali siano le loro classi concrete.

\paragraph{Problema} Si consideri lo sviluppo di un toolkit per la realizzazione di GUI in grado di supportare diversi look-and-feel. Affinché sia possibile il codice che la implementa non deve dipendere dal tipo specifico dei widget utilizzati quindi non può istanziarli direttamente.

\begin{figure}[H]
    \centering
    \includegraphics[width=1\linewidth]{assets/pattern/abstract-factory/abstract-factory-esempio.png}
\end{figure}

\paragraph{Soluzione} L’interfaccia WidgetFactory introduce un metodo per la creazione di ciascun tipo di base di widget definito a sua volta da un’oportuna interfaccia. I client invocano i metodi definiti da WidgetFactory per ottenere istanze di widget senza conoscere la classe concreta che utilizzano. Esiste una classe concreta che implementa WidgetFactory per ciascuno dei L\&F considerati.

\begin{figure}[H]
    \centering
    \includegraphics[width=1\linewidth]{assets/pattern/abstract-factory/abstract-factory-struttura.png}
\end{figure}

\paragraph{Struttura e Conseguenze} I partecipanti del pattern sono:
\begin{itemize}
    \item \textbf{AbstractFactory} (WidgetFactory): dichiara un’interfaccia per le operazioni di creazione di oggetti prodotto astratti.
    \item \textbf{ConcreteFactory} (MotifWidgetFactory, GtkWidgetFactory): implementa le operazioni degli oggetti prodotto concreti. 
    \item \textbf{AbstractProduct} (Window, ScrollBar): dichiara un’interfaccia per un tipo di prodotti
    \item \textbf{ConcreteProduct} (MotifWindow, GtkWindow): implementa l’interfaccia Product definendo un oggetto prodotto creato dalla corrispondente factory concreta.
\end{itemize}

Il pattern AbstractFactory consente quindi di isolare le classi concrete (processo di creazione incapsulato nella Factory), cambiare agilmente la famiglia di prodotti utilizzata (cambio di configurazione cambiando il tipo di Factory) e promuovere la coerenza nell'utilizzo dei prodotti.

È bene notare che l'aggiunta del supporto a nuove tipologie di prodotti è difficile in quanto comporta la modifica dell'interfaccia AbstractFactory e, di conseguenza, di tutte le classi che la implementano.

In più è possibile implementare il Factory come Singleton (\ref{singleton})

\begin{figure}[H]
    \centering
    \includegraphics[width=1\linewidth]{assets/pattern/abstract-factory/abstract-factory-sequence.drawio.png}
    \caption{Sequence Diagram di AbstractFactory}
\end{figure}

\newpage
\subsection{Builder}
\label{builder}

\textbf{Scopo}: Creazionale \\
\textbf{Raggio d'azione}: Oggetti

\paragraph{Definizione} Il patter Builder permette di separare la costruzione di un oggetto complesso dalla sua rappresentazione, in modo che lo stesso processo di costruzione possa essere utilizzato per creare rappresentazioni diverse.

\paragraph{Motivazione} Prendiamo in considerazione un’applicazione capace di leggere documenti in formato RTF che può supportare la conversione in altri formati (ASCII, LaTeX).

\begin{figure}[H]
    \centering
    \includegraphics[width=0.75\linewidth]{assets/pattern/builder/builder-esempio.png}
\end{figure}

Una soluzione consiste nel configurare la classe RTFReader con un oggetto conforme all’interfaccia TextConverter in grado di gestire la conversione in un altro formato. Il documento nel formato di uscita viene costruito man mano che gli elementi del documento RTF sono analizzati.

\begin{figure}[H]
    \centering
    \includegraphics[width=0.75\linewidth]{assets/pattern/builder/builder-struttura.png}
    \caption{Class Diagram del pattern Builder}
\end{figure}

\paragraph{Applicabilità} È consigliabile utilizzare il pattern Builder quando:
\begin{itemize}
    \item L'algoritmo di creazione di un oggetto complesso è indipendente dalle parti che lo costituiscono.
    \item Il processo di costruzione deve permettere rappresentazioni diverse pper l'oggetto costruito.
\end{itemize}

\paragraph{Struttura} Il pattern è composto da:
\begin{itemize}
    \item \textbf{Builder}: specifica l’interfaccia astratta che crea le parti dell’oggetto Product (Alternativa step-by-step ad AbstractFactory \ref{abstract-factory}). 
    \item \textbf{ConcreteBuilder}: costruisce e assembla le parti del prodotto implementando l’interfaccia Builder; definisce e tiene traccia della rappresentazione che crea.
    \item \textbf{Director}: costruisce un oggetto utilizzando l’interfaccia Builder.
    \item \textbf{Product}: rappresenta l’oggetto complesso e include le classi che definiscono le parti che lo compongono, includendo le interfacce per assemblare le parti nel risultato finale. (Spesso il builder è usato per costruire Composite \ref{composite})
\end{itemize}

\begin{figure}[H]
    \centering
    \includegraphics[width=0.75\linewidth]{assets/pattern/builder/builder-sequence.drawio.png}
    \caption{Sequence Diagram del pattern Builder}
\end{figure}

\begin{enumerate}
    \item Il Client create un oggetto Director e lo configura con il Builder desiderato. 
    \item Il Director notifica il Builder quando una parte di un prodotto deve essere costruita.
    \item Il Builder gestisce la richiesta e aggiunge le parti richieste al Product.
    \item Il Client richiede l'oggetto al Builder.
\end{enumerate}

\paragraph{Conseguenze} Il pattern Builder consente quindi di:
\begin{itemize}
    \item \textbf{Variare la rappresentazione interna di un prodotto}: nasconde la rappresentazione e la struttura interna del prodotto. Nasconde anche come il prodotto viene assemblato.
    \item \textbf{Isolare il codice per la costruzione la rappresentazione}: Ogni ConcreteBuilder contiene tutto il codice necessario a creare ed assemblare un particolare tipo di prodotto. Diversi Director possono riusare lo stesso Builder per creare varianti di Product.
    \item \textbf{Avere maggiore controllo del processo di costruzione}: il Builder costruisce iol prodotto step-by-step sotto la supervisione del Director.
    \item \textbf{Risolvere il problema dei \textit{costruttori telescopici}} :si propone come alternativa a tecnologie come JavaBeans.
\end{itemize}



\newpage

\textbf{Esempio Java}

\begin{minted}[
    fontsize=\footnotesize,
    linenos,
]{java}
public class NutritionFacts { 

    private final int servingSize; 
    private final int servings; 
    private final int calories ; 
    private final int fat ; 
    private final int sodium; 
    private final int carbohydrate; 
    
    public static class Builder { 
        // Required parameters 
        private final int servingSize; 
        private final int servings; 
        
        // Optional 
        private int calories = 0; 
        private int fat = 0; 
        private int carbohydrate = 0; 
        private int sodium = 0; 
        
        public Builder( int servingSize, int servings) { 
            this.servingSize = servingSize; 
            this.servings = servings; 
        } 
        
        public Builder calories ( int val ) { 
            calories = val ; return this;
        } 
        
        public Builder fat ( int val ) { 
            fat = val ; 
            return this; 
        } 
        
        public Builder carbohydrate(int val ) {
            carbohydrate = val; 
            return this; 
        } 
        
        public Builder sodium(int val) { 
            sodium = val; 
            return this; 
        } 
        
        public NutritionFacts build () { 
            return new NutritionFacts(this); 
        } 
    }
    
    private NutritionFacts (Builder builder ) { 
        servingSize = builder.servingSize; 
        servings = builder.servings; 
        calories = builder.calories; 
        fat = builder.fat;
        sodium = builder.sodium; 
        carbohydrate = builder.carbohydrate;
    }
}

// Utilizzo:
NutritionFacts cocaCola = new NutritionFacts.Builder(240, 8)
                            .calories(100)
                            .sodium(35)
                            .carbohydrate(27)
                            .build();

\end{minted}


\newpage
\subsection{Factory Method (\textit{o Virtual Constructor})}
\label{factory-method}

\textbf{Scopo}: Creazionale  \\
\textbf{Raggio d'azione}: Classi

\paragraph{Definizione} Definisce un'interfaccia per la creazione di un oggetto, lasciando alle sottoclassi la decisione sulla classe concreta che sarà istanziata. Consente di delegare la creazione di un oggetto alle sottoclassi.

\paragraph{Motivazione} Si consideri un framework per applicazioni in grado di presentare più documenti agli utenti.

\begin{figure}[H]
    \centering
    \includegraphics[width=0.75\linewidth]{assets/pattern/factory-method/factory-method-esempio.png}
\end{figure}

La classe Application è responsabile della gestione di oggetti conformi all’interfaccia Document e della loro creazione. Application sa solo quando dovrà creare un nuovo documento ma non di che tipo né come dovrà farlo. La responsabilità è spostata all’esterno del framework: le sottoclassi di Application devono fornire l’implementazione del metodo factory \textit{createDocument()} in modo da restituire un’istanza della classe appropriata

\paragraph{Applicabilità} È consigliabile utilizzare il pattern FactoryMethod quando:
\begin{itemize}
    \item Una classe non può prevedere la classe dell'oggetto che deve essere creato
    \item Una classe vuole che le sue sottoclassi specifichino gli oggetti da creare
    \item Si vuole delegare (e localizzare) la resposabilità a varie sottoclassi (helpers).
\end{itemize}

\begin{figure}[H]
    \centering
    \includegraphics[width=0.75\linewidth]{assets/pattern/factory-method/factory-method-struttura.png}
    \caption{Class Diagram del pattern Factory Method}
\end{figure}

\paragraph{Struttura} Il pattern è composto da:
\begin{itemize}
    \item \textbf{Product} (Document): definisce l'interfaccia degli oggetti creati dal metodo factory;
    \item \textbf{ConcreteProduct} (MyDocument): implementa l'interfaccia Product;
    \item \textbf{Creator} (Application): dichiara il metodo factory che restituisce un oggetto di tipo Product. Può invocare \textit{createProduct()} per creare un prodotto;
    \item \textbf{ConcreteCreator} (MyApplication): sovrascrive \textit{createProduct()} in modo da restituire una specifica istanza di ConcreteProduct;
\end{itemize}

Il Creator fa riferimento alle sue sottoclassi per definire il metodo Factory che restituisce un'istanza appropriata di ConcreteProduct. 

Gli utilizzatori potrebbero essere costretti a definire sottoclassi di Creator per creare un particolare oggetto ConcreteProduct.

\begin{figure}[H]
    \centering
    \includegraphics[width=0.5\linewidth]{assets/pattern/factory-method/factory-method-sequence.drawio.png}
    \caption{Sequence Diagram del pattern Factory Method}
\end{figure}

\paragraph{Conseguenze} Il pattern FactoryMethod consente quindi di:
\begin{itemize}
    \item \textbf{Fornisce un punto d’aggancio per le sottoclassi} per la produzione di una versione specializzata di un prodotto.
    \item \textbf{Connette gerarchie di classi parallele}: il metodo factory potrebbe invocato da oggetti diversi dal Creator.
    \item Elimina la necessità di riferirsi a classi dipendenti dall’applicazione all’interno del codice del framework.
\end{itemize}

\begin{figure}[H]
    \centering
    \includegraphics[width=0.75\linewidth]{assets/pattern/factory-method/factory-method-parallelo.png}
    \caption{Gerarchie di classi parallele}
\end{figure}

\newpage
\subsection{Prototype}
\label{prototype}

\textbf{Scopo}: Creazionale \\
\textbf{Raggio d'azione}: Oggetti

\paragraph{Definizione} Specifica la tipologia di oggetti da creare utilizzando un'istanza prototipo e creare nuovi oggetti clonando questo prototipo.

\paragraph{Motivazione} Si pensi ad applicazione che consenta rappresentare degli elementi grafici nel piano cartesiano. L’applicazione potrebbe avere una barra di tasti per effettuare varie operazioni. Un’azione tipica è quella di creare un nuovo oggetto grafico. Per inserire differenti oggetti l’azione da compiere e` identica a parte il tipo di oggetto da creare.

\begin{figure}[H]
    \centering
    \includegraphics[width=0.75\linewidth]{assets/pattern/prototype/prototype-esempio.png}
    \caption{Esempio del pattern Prototype}
\end{figure}

Una soluzione consiste nel configurare l’oggetto responsabile della creazione (il tasto) con un prototipo dell’oggetto da creare.

\paragraph{Applicabilità} È consigliabile utilizzare il pattern Prototype quando:
\begin{itemize}
    \item Le classi da istanziare sono specificate a run-time
    \item Si vuole evitare di costruire una gerarchia di classi Factory
    \item Le istanze di una classe possono avere una o poche combinazioni differenti di stati
\end{itemize}

\begin{figure}[H]
    \centering
    \includegraphics[width=0.75\linewidth]{assets/pattern/prototype/prototype-struttura.png}
    \caption{Class Diagram del pattern Prototype}
\end{figure}

\paragraph{Struttura} Il pattern è composto da:
\begin{itemize}
    \item \textbf{Prototype} (Graphic): specifica l'interfaccia che consente la clonazione
    \item \textbf{ConcretePrototype} (Staff, WholeNote, HalfNote): implementa l'operazione di clonazione
    \item \textbf{Client} (GraphicTool): crea un nuovo oggetto chiedendo ad un prototipo di clonarsi
\end{itemize}

Il Client richiede a Prototype un clone di se stesso.


\paragraph{Conseguenze} Condivide molte delle conseguenze di AbstractFactory (\ref{abstract-factory}) e Builder (\ref{builder}): nasconde le classi dei prodotti concreti, riducendo il numero di classi che devono essere note al client. 

È utile precisare che i design che fanno utilizzo di Composite (\ref{composite}) e Decorator (\ref{decorator}) possono beneficiare di Prototype.

In più:
\begin{itemize}
    \item Consente di aggiungere e rimuovere prodotti durante l’esecuzione;
    \item Consente la specifica di nuovi oggetti variando i valori;
    \item Consente la specifica di nuovi oggetti variando la struttura;
    \item Riduce il numero di sottoclassi
    \item Obbliga l'implementazione dell’operazione \textit{clone()}
\end{itemize}

\newpage

\paragraph{Esempio Java} Utilizzo dell'interfaccia \textit{Cloneable}, è possibile clonare oggetti in modo superficiale (shallow copy) o in maniera profonda (deep copy) ridefinendo il metodo clone().

\begin{minted}[
    fontsize=\footnotesize,
    linenos,
]{java}
// Shallow copy
public class Point2D implements Cloneable {
    private double x; 
    private double y;
    
    @Override public Point2D clone() { 
        try { 
            Point2D clone = (Point2D) super.clone(); 
            return clone;
        } catch (CloneNotSupportedException e) {
            throw new Error(e);
        } 
    } 
}
\end{minted}

\newpage

\begin{minted}[
    fontsize=\footnotesize,
    linenos,
]{java}

// Deep copy
public abstract class PolinomioAstratto implements Polinomio, Cloneable { 
    protected abstract PolinomioAstratto getPrototype(); 

    @Override public Polinomio add(Polinomio p) { 
        // crea un nuovo polinomio 
        Polinomio somma = getPrototype().clone(); 

        // aggiunge ciascun monomio di this al polinomio somma 
        for (Monomio m : this) somma.add(m); 
        
        // aggiunge ciascun monomio di p al polinomio somma 
        for (Monomio m : p) somma.add(m); 
        
        return somma; 
    } 
    
    @Override public PolinomioAstratto clone() { 
        try { 
            return (PolinomioAstratto) super.clone(); 
        } catch (CloneNotSupportedException e) { 
            throw new Error(e); 
        }
    }
}

public class PolinomioLL extends PolinomioAstratto { 
    private static PolinomioLL prototype; 
    private LinkedList<Monomio> monomi = new LinkedList<>(); 
    
    @Override protected synchronized PolinomioLL getPrototype() { 
        if ( prototype==null ) prototype = new PolinomioLL(); 
        return prototype;
    } 
    
    @Override public PolinomioLL clone() {
        PolinomioLL p = (PolinomioLL) super.clone();
        p.monomi = new LinkedList<Monomio>(); 
        for (Monomio m : this) p.monomi.add(m); 
        return p;
    }
}

\end{minted}

\newpage
\subsection{Singleton}
\label{singleton}

\textbf{Scopo}: Creazionale \\
\textbf{Raggio d'azione}: Oggetti

\paragraph{Definizione} Il pattern Singleton assicura che un classe abbia una sola istanza e fornisca un solo punto di accesso globale a tale istanza.

\paragraph{Motivazione} In un sistema potrebbero esistere più stampanti, ma potrebbe essere presente soltanto una coda di stampa. In un sistema operativo dovrebbe essere presente solo un file system e un solo window manager

Per assicurare che una classe abbia una sola istanza e che tale istanza sia facilmente accessibile per gli utilizzatori si può fare in modo che la classe stessa abbia la responsabilità di creare le proprie istanze. La classe può assicurare che nessun’altra istanza possa essere creata e può fornire un modo semplice per accedere all’istanza.

\paragraph{Applicabilità} È consigliabile utilizzare il pattern Singleton quando:
\begin{itemize}
    \item Deve esistere esattamente un’istanza di una classe resa accessibile ai client attraverso un punto di accesso noto a tutti gli utilizzatori.
    \item L’unica istanza deve poter essere estesa attraverso la definizione di sottoclassi ed i client devono essere in grado di utilizzare le istanze estese senza dover modificare il proprio codice.
\end{itemize}

\begin{figure}[H]
    \centering
    \includegraphics[width=0.4\linewidth]{assets/pattern/singleton/singleton-struttura.png}
    \caption{Class Diagram del pattern Singleton}
\end{figure}

\paragraph{Struttura} Il pattern è composto da:
\begin{itemize}
    \item \textbf{Singleton}: definisce un'operazione statica \textit{Instance()} che permette ai client di accedere alla sua unica istanza (della cui creazione è responsabile).
\end{itemize}

\paragraph{Conseguenze} Il pattern Singleton consente quindi di:
\begin{itemize}
    \item Controllare l'accesso all'istanza unica.
    \item Ridurre il numero di variabili (name space)
    \item Rifinire operazioni e rappresentazione
\end{itemize}

\newpage

\textbf{Java}

\begin{minted}[
    fontsize=\footnotesize,
    linenos,
]{java}
public final class Singleton { 
    private static Singleton INSTANCE = null; 
    
    private Singleton(){} 
    
    public static synchronized Singleton getInstance() { 
        if (INSTANCE == null) { 
            INSTANCE = new Singleton(); 
        } 
        return INSTANCE; 
    } 
}
\end{minted}

\newpage

\section{Strutturali}

I pattern strutturali si occupano di come classi e oggetti sono composti per formare strutture più grandi. 
\begin{itemize}
    \item \textbf{Class Structural}: usano l'ereditarietà per definire interfacce o implementazioni;
    \item \textbf{Object Structural}: descrivono modi per comporre oggetti per realizzare nuove funzionalità
\end{itemize}

È bene sottolineare che alcuni pattern fanno utilizzo di ereditarietà multipla, non sempre applicabile in linguaggi come Java, che permettono di implementare molteplici interfacce, ma estendere una sola classe (astratta o concreta).

\subsection{Adapter (\textit{o Wrapper})}
\label{adapter}

\textbf{Scopo}: Strutturale \\
\textbf{Raggio d'azione}: Classi, Oggetti

\paragraph{Definizione} Permette di convertire l'interfaccia di una classe in un'altra interfaccia richiesta dal client.Consente a classi diverse di cooperare quando ciò non sarebbe possibile a causa di interfacce incompatibili.

\paragraph{Motivazione} A volte una classe preesistente, progettata per essere riutilizzata, non può essere riusata perché incompatibile con l'interfaccia richiesta dall'applicazione. Nell’esempio, la classe XXXTriangle non può essere riusata dove ci si aspetta l’interfaccia Figure2D perché non è compatibile con essa.

\begin{multicols}{2}
\begin{figure}[H]
    \centering
    \includegraphics[width=1\linewidth]{assets/pattern/adapter/adapter-esempio-class.png}
    \caption{Esempio di Class Adapter}
\end{figure}
\columnbreak
\begin{figure}[H]
    \centering
    \includegraphics[width=1\linewidth]{assets/pattern/adapter/adapter-esempio-object.png}
    \caption{Esempio di Object Adapter}
\end{figure}
\end{multicols}

Pensare di modificare XXXTriangle per renderla conforme a Figure2D non è una buona soluzione (si legherebbe al contesto specifico e il codice sorgente potrebbe non essere disponibile).

Si introduce una classe Triangle che sia al contempo erede di entrabe XXXTriangle e Figure2D. L'implementazione dei metodi di Figure2D sfrutta quelli di XXXTriangle.

È possibile introdurre la classe Triangle anche di modo che faccia riferimento ad un oggetto (istanza di XXXTriangle) e vada ad implementare l'interfaccia Figure2D delegando l'esecuzione all'oggetto incapsulato.

\begin{figure}[H]
    \centering
    \includegraphics[width=0.75\linewidth]{assets/pattern/adapter/adapter-sequence.drawio.png}
    \caption{Sequence diagram raggio d'azione}
\end{figure}

\paragraph{Applicabilità} È consigliabile utilizzare il pattern Adapter quando:
\begin{itemize}
    \item Si vuole utilizzare una classe esistente e la sua interfaccia non è compatibile con quella che serve;
    \item Si vuole creare una classe riusabile che coopera con classi non correlate o non conosciute;
    \item Si vogliono usare varie sottoclassi esistenti, ma non sarebbe pratico adattare le loro interfacce tramite ereditarietà (solo per object);
\end{itemize}

\paragraph{Struttura} Il pattern è composto dai seguenti partecipanti:
\begin{itemize}
    \item \textbf{Target} (Figure2D): definisce l'interfaccia specifica del dominio utilizzata dal client
    \item \textbf{Client}: collabora con oggetti compatibili con l'intefaccia Target
    \item \textbf{Adaptee} (XXXTriangle): individua un'interfaccia che deve essere adattata.
    \item \textbf{Adapter} (Triangle): adatta l'interfaccia Adaptee al'interfaccia Target
\end{itemize}

I client invocano le operazioni su un'istanza di Adapter, il quale invoca operazioni di Adaptee per soddisfare la richiesta

\begin{figure}[H]
    \centering
    \includegraphics[width=0.75\linewidth]{assets/pattern/adapter/adapter-struttura-class.png}
    \caption{Class Diagram di Adapter (Class)}
\end{figure}

\begin{figure}[H]
    \centering
    \includegraphics[width=0.75\linewidth]{assets/pattern/adapter/adapter-struttura-object.png}
    \caption{Class Diagram di Adapter (Object)}
\end{figure}

\paragraph{Conseguenze} Il pattern Builder consente quindi di:
\begin{itemize}
    \item Scegliere lo spettro di operazioni da supportare (o wrappare);
    \item Supportare i \textbf{pluggable adapter}: classi che supportano l'adattamento di interfaccia
    \item Usare i \textbf{two-way adapters}: permettono all'Adapter di supportare operazioni dell'Adaptee e viceversa.
\end{itemize}

\textbf{Class Adapter}
\begin{itemize}
    \item Adatta l'interfaccia Adaptee all'interfaccia Target basandosi su una classe concreta.
    \item Non può essere utilizzata se Adaptee è astratta oppure è un'interfaccia.
    \item Consente ad Adapter di sovrascrivere parte del comportamento di Adaptee essendo una sua sottoclasse
    \item Non occorrono ulteriori indirezioni per raggiungere l'oggetto adattato
\end{itemize}

\textbf{Object Adapter}
\begin{itemize}
    \item Permette ad un singolo Adapter di operare con Adaptee e le sue sottoclassi, se esistono. Può in tal caso aggiungere funzionalità a tutti gli Adaptee
    \item Rende difficile sovrascrivere il comportamento di Adaptee non essendo una sua sottoclasse
    \item Aggiunge un livello di indirezione per raggiungere l'oggetto adattato.
\end{itemize}

\newpage
\subsection{Bridge (\textit{o Handle, Body})}


\textbf{Scopo}: Strutturale \\
\textbf{Raggio d'azione}: Oggetti

\paragraph{Definizione} Disaccoppia un’astrazione dalla sua implementazione in modo che le due possano variare indipendentemente l’una dall’altra.

\paragraph{Problema}

\paragraph{Soluzione} 

\paragraph{Struttura e Conseguenze} 

\newpage
\subsection{Composite}
\label{composite}

\textbf{Scopo}: Strutturale \\
\textbf{Raggio d'azione}: Oggetti

\paragraph{Definizione} Il pattern Composite permette di comporre oggetti in strutture ad albero per rappresentare gerarchie \textit{parte-tutto} e consentire ai client di trattare oggetti singoli e composizioni in modo uniforme.

\paragraph{Problema} Applicazioni quali editor grafici vettoriali o ambienti per la progettazione di circuiti consentono agli utenti di costruire diagrammi complessi a partire da semplici componenti. Componenti semplici possono essere raggruppati per costruire componenti più complessi che possono, a loro volta, essere utilizzati come parti di componenti ancor più complessi. Solitamente si introducono alcune classi per modellare gli oggetti semplici ed altre per rappresentare gli oggetti ottenuti per composizione.

\begin{figure}[H]
    \centering
    \includegraphics[width=0.8\linewidth]{assets/pattern/composite/composite-esempio.png}
\end{figure}

\paragraph{Soluzione} Il pattern Composite introduce un’interfaccia comune per gli oggetti semplici e per quelli compositi in modo che il codice cliente li possa trattare uniformemente. Nell’esempio, le classi Line, Rectangle e Text definiscono gli oggetti primitivi. Il metodo \textit{draw()} implementa l’operazione di disegno. Poichè gli oggetti primitivi non hanno figli, le operazioni per la gestione dei componenti non sono implementate. La classe Picture è un aggregato di oggetti Graphic ed implementa \textit{draw()} invocandolo a sua volta sugli oggetti da cui è composto.

\begin{figure}[H]
    \centering
    \includegraphics[width=0.5\linewidth]{assets/pattern/composite/composite-object.png}
    \caption{Tipica struttura di oggetti Graphic ricorsivamente composti}
\end{figure}

\newpage

\begin{figure}[H]
    \centering
    \includegraphics[width=1\linewidth]{assets/pattern/composite/composite-struttura.png}
\end{figure}

\paragraph{Struttura e Conseguenze} Il pattern composite è composto da:
\begin{itemize}
    \item \textbf{Component} (Graphic): dichiara l’interfaccia per gli oggetti componibili. Può essere una classe astratta che fornisce un’implementazione di default per i metodi di gestione dei componenti figli. Introduce (opzionale) un metodo per accedere al componente genitore. 
    \item \textbf{Leaf} (Rectangle,Line, etc.): rappresenta gli oggetti primitivi i quali non hanno figli. 
    \item \textbf{Composite} (Picture): definisce il comportamento dei componenti che hanno figli. Memorizza i componenti figli e implementa i relativi metodi introdotti dall’interfaccia Component.
\end{itemize}

\begin{figure}[H]
    \centering
    \includegraphics[width=1\linewidth]{assets/pattern/composite/composite-activity.drawio.png}
\end{figure}

\newpage
\subsection{Decorator (\textit{o Wrapper})}
\label{decorator}

\textbf{Scopo}: Strutturale \\
\textbf{Raggio d'azione}: Oggetti

\paragraph{Definizione} Il pattern decorator permetter di aggiungere dinamicamente responsabilità ad un oggetto. Fornisce un'alternativa flessibile all'uso dell'ereditarietà come strumento per l'estensione delle funzionalità.

\paragraph{Moticazione} Talvolta è necessario aggiungere responsabilità ad un singolo oggetto e non ad un intera classe. Si consideri una libreria per la realizzazione di interfacce utente che deve permettere di aggiungere bordi o altri elementi a ciascun componente grafico. Ricorrere all'ereditarietà complicherebbe la cosa, andrebbero create sottoclassi per ogni componente aggiuntivo, in più renderebbe difficile comporre più estensioni di comportamento.

\begin{figure}[H]
    \centering
    \includegraphics[width=0.5\linewidth]{assets/pattern/decorator/decorator-esempio-grafico.png}
\end{figure}

La soluzione consiste nel racchiudere il componente da decorare dentro un altro che sarà responsabile di disegnare il bordo o aggiungere altri componenti visivi. L'oggetto contenitore è detto \textit{decoratore} ed ha la stessa interfaccia dell'oggetto decorato così da poter essere trasparente al client. È possibile che svolga azioni aggiuntive prima o dopo aver trasferito la richiesta all'oggetto decorato.

\begin{figure}[H]
    \centering
    \includegraphics[width=0.75\linewidth]{assets/pattern/decorator/decorator-esempio.png}
    \caption{Esempio di utilizzo del pattern Decorator}
\end{figure}

Nell'esempio l'interfaccia VisualComponent definisce il tipo generico di componenti visuali. La classe TextView consente di visualizzare del testo in una finestra. La classe astratta Decorator inoltra semplicemente le richieste al componente incapsulato. BorderDecorator e ScrollDecorator consentono rispettivamente di aggiungere bordi e barre di scorrimento.

\paragraph{Applicabilità} È consigliabile utilizzare il pattern Decorator quando:
\begin{itemize}
    \item Si vogliono aggiungere responsabilità ad un oggetto dinamicamente e in modo trasparente, senza influenzare altri oggetti;
    \item Si vuole supportare un largo numero di estensioni senza riccorrere all'ereditarietà (che farebbe esplodere il numero di sottoclassi);
\end{itemize}

\begin{figure}[H]
    \centering
    \includegraphics[width=0.75\linewidth]{assets/pattern/decorator/decorator-object-diagram.png}
    \caption{Object Diagram del pattern Decorator}
\end{figure}

\paragraph{Struttura} Il pattern è composto dai seguenti partecipanti:
\begin{itemize}
    \item \textbf{ServiceIF} (VisualComponent): definisce l'interfaccia comune per gli oggetti ai quali possono essere aggiunte nuove responsabilità dinamicamente.
    \item \textbf{ConcreteService} (TextView): definisce un oggetto al quale possono essere aggiunte ulteriori responsabilità
    \item \textbf{Decorator}: mantiene un riferimento ad un oggetto di tipo ServiceIF e, al contempo implementa l'interfaccia ServiceIF.
    \item \textbf{ConcreteDecorator} (ScrollDecorator, BorderDecorator): aggiunge responsabilità al componente
\end{itemize}

\begin{figure}[H]
    \centering
    \includegraphics[width=0.75\linewidth]{assets/pattern/decorator/decorator-struttura.png}
    \caption{Class Diagram del pattern Decorator}
\end{figure}

\paragraph{Conseguenze} Il pattern Decorator consente quindi di:
\begin{itemize}
    \item Avere maggiore flessibilità rispetto all'ereditarietà;
    \item Poter utilizzare differenti combinazioni di oggetti attraverso l'utilizzo di diversi decoratori;
\end{itemize}
Inoltre:
\begin{itemize}
    \item L'uso nei progetti porta a sistemi composti di molti oggetti simili interconnessi, facile da interpretare dal progettista, meno da esterni;
    \item Decoratore e oggetto decorato sono uguali in termini di comportamento ma differiscono in termini di identità
    \item Bisogna porre attenzione in come si compongono i decoratori, per esempio evitare dipendenze circolari.
\end{itemize}

\begin{figure}[H]
    \centering
    \includegraphics[width=0.75\linewidth]{assets/pattern/decorator/decorator-sequence.drawio.png}
    \caption{Sequence Diagram del pattern Decorator}
\end{figure}


\newpage
\subsection{Façade}
\label{facade}

\textbf{Scopo}: Strutturale \\
\textbf{Raggio d'azione}: Oggetti

\paragraph{Definizione} Il pattern Façade fornisce un'interfaccia unificata per un insieme di interfacce presenti in un sottosistema. Definisce un'interfaccia di livello più alto che rende il sistema più semplice da utilizzare.

\begin{figure}[H]
    \centering
    \includegraphics[width=1\linewidth]{assets/pattern/facade/facade-struttura.png}
    \caption{Class Diagram del pattern Façade}
\end{figure}

\paragraph{Struttura e Conseguenze} Il pattern Façade è composto da:
\begin{itemize}
    \item \textbf{Façade} (MessageCreator): conosce le classi nel sottosistema che sono responsabili di gestire una richiesta. 
    \item \textbf{Classi del sottosistema} (Message,MessageBody, Attachment, etc.): Implementano le funzionalità del sottosistema. Non hanno alcuna conoscenza dell’esistenza del Façade: non hanno alcun riferimento ad esso.
\end{itemize}

\begin{figure}[H]
    \centering
    \includegraphics[width=1\linewidth]{assets/pattern/facade/facade-problema.png}
    \caption{Problema: creazione di un messaggio e-mail}
\end{figure}

\begin{figure}[H]
    \centering
    \includegraphics[width=1\linewidth]{assets/pattern/facade/facade-soluzione.png}
    \caption{Soluzione con pattern Façade}
\end{figure}

Il pattern permette quindi di nascondere ai client i componenti del sottosistema, riducendo il numero degli oggetti con cui i client interagiscono. Promuove il \textbf{basso accoppiamento} tra sottosistema e client, e \textbf{alta coesione} interna. Riduce le dipendenze di compilazione e non impedisce l'utilizzo delle classi del sottosistema qualora fosse necessario.

È utile sapere che un'interfaccia che implementa il pattern Façade può facilmente finire per essere accoppiata a tuttle le classi di un applicativo.

\newpage
\subsection{Flyweight}
\label{flyweight}

\textbf{Scopo}: Strutturale \\
\textbf{Raggio d'azione}: Oggetti

\paragraph{Definizione} Il pattern Flyweight permette di utilizzare la condivisione per supportare in modo efficiente un gran numero di oggetti a granularità fine.

\begin{figure}[H]
    \centering
    \includegraphics[width=1\linewidth]{assets/pattern/flyweight/flyweight-problema.png}
    \caption{Problema del particle system}
\end{figure}

\paragraph{Problema} Si consideri ad esempio un semplice videogioco in cui si è scelto di implementare un sistema particellare realistico e renderlo una caratteristica distintiva. Grandi quantità di proiettili, missili e schegge di esplosioni dovrebbero volare su tutta la mappa di gioco. Ogni particella è rappresentata da un oggetto separato contenente molti dati. Durante l’esecuzione, ad un certo punto, non c'è più spazio a sufficienza nella RAM per le particelle appena create, quindi il programma va in crash.

\begin{figure}[H]
    \centering
    \includegraphics[width=1\linewidth]{assets/pattern/flyweight/flyweight-soluzione.png}
    \caption{Soluzione tramite pattern Flyweight}
\end{figure}

\paragraph{Soluzione} Il pattern Flyweight descrive come condividere oggetti in modo da consentire il loro uso a granularità fine senza avere costi proibitivi. Si costruisce un oggetto condiviso che può essere usato simultaneamente in più contesti. Data la loro natura c'è bisogno di distinguere tra stato interno (o intrinseco, informazioni indipendenti dal contesto) e stato esterno (o estrinseco, informazioni dipendenti dal contesto). Lo stato esterno viene passato dal client.

\begin{minipage}{0.5\linewidth}
    \paragraph{Esempio} Nell’esempio i campi color e sprite consumano molta più memoria rispetto agli altri. Inoltre essi memorizzano informazioni pressoché identiche tra le particelle. Ad esempio tutti i proiettili avranno lo stesso colore e lo stesso sprite. Gli altri campi, quali le coordinate, il vettore direzione e la velocità hanno valori distinti per ogni particella e inoltre cambiano con il tempo. Colore e sprite corrispondono allo stato intrinseco mentre gli altri campi allo stato estrinseco. La classe Particle modella lo stato intrinseco (immutabile), la classe MovingParticle quello estrinseco. Solo tre oggetti diversi saranno sufficienti a rappresentare lo stato esterno di tutte le particelle del gioco: uno per i proiettili, uno per il missili e uno per le schegge. Nell’esempio lo stato intrinseco è memorizzato nell’array particle della classe Game mentre quello estrinseco nell’array mps. Una soluzione più elegante consiste nell’introdurre una classe di contesto che memorizza lo stato estrinseco e un riferimento all’oggetto flyweight che corrisponde a quello intrinseco.
\end{minipage}
\hfill
\begin{minipage}{0.5\linewidth}
    \includegraphics[width=1\linewidth]{assets/pattern/flyweight/flyweight-esempio.png}
\end{minipage}

\begin{figure}[H]
    \centering
    \includegraphics[width=1\linewidth]{assets/pattern/flyweight/flyweight-struttura.png}
    \caption{Class Diagram del pattern Flyweight}
\end{figure}

\paragraph{Struttura e Conseguenze} Il pattern è composto da:
\begin{itemize}
    \item \textbf{Flyweight}: dichiara un’interfaccia attraverso la quale gli oggetti flyweight possono ricevere lo stato esterno e agire di conseguenza.
    \item \textbf{FlyweightFactory}: crea e gestisce gli oggetti flyweight. Si assicura che i flyweight siano condivisi in modo appropriato. Quando un client richiede un flyweight, l’oggetto FlyweightFactory restituisce un’istanza esistente, oppure, se non esiste alcuna istanza, prima la crea e poi la restituisce.
    \item \textbf{Context}: contiene lo stato estrinseco, unico tra tutti gli oggetti originali. Quando un oggetto context è accoppiato con uno degli oggetti flyweight, rappresenta l’intero stato di un oggetto originale.
    \item \textbf{Client}: calcola oppure memorizza lo stato estrinseco dei flyweight. Dal punto di vista del client, un flyweight è un oggetto template che può essere configurato a tempo di esecuzione fornendogli dati contestuali come parametri dei suoi metodi.
\end{itemize}

\newpage
\subsection{Proxy (\textit{o Surrogate})}
\label{proxy}

\textbf{Scopo}: Strutturale \\
\textbf{Raggio d'azione}: Oggetti

\paragraph{Definizione} Fornisce un surrogato o segnaposto di un altro oggetto per controllare l'accesso a tale oggetto.

\paragraph{Motivazione} Si consideri un editor che consente di rappresentare anche immagini all’interno dei documenti. Per velocizzare il caricamento in memoria dei documenti può essere opportuno ritardare il caricamento delle immagini fino a quando non è necessario visualizzarle.

\begin{figure}[H]
    \centering
    \includegraphics[width=0.75\linewidth]{assets/pattern/proxy/proxy-esempio.png}
    \caption{Esempio di utilizzo del pattern}
\end{figure}

La soluzione consiste nell'utilizzare un oggetto surrogato dell’immagine, ad esempio che pre-occupi lo stesso spazio.

\paragraph{Applicabilità} Il pattern Proxy è utilizzabile nelle seguenti situazioni:
\begin{itemize}
    \item \textbf{Remote Proxy}: fornisce una rappresentazione di un oggetto in un diverso address-space;
    \item \textbf{Virtual Proxy}: crea oggetti costosi su richiesta;
    \item \textbf{Protection Proxy}: controlla l'accesso all'oggetto originale;
    \item \textbf{Smart Reference}: è un sostituito di un puntatore classico che esegue ulteriori operazioni quando un oggetto viene utilizzato;
\end{itemize}

\paragraph{Struttura} Il pattern è composto da:
\begin{itemize}
    \item \textbf{Proxy} (ImageProxy): mantiene un riferimento all'oggetto di tipo RealSubject (di cui è \textit{surrogato}). Ha la stessa interfaccia di Subject. Controlla l'accesso all'oggetto rappresentato e può essere responsabile della sua creazione o eliminazione;
    \item \textbf{Subject} (Graphic): definisce l'interfaccia comune per RealSubject e Proxy consentendo di usare Proxy ove ci si attende un RealSubject
    \item \textbf{RealSubject} (Image): definisce l'oggetto reale rappresentato dal proxy;
\end{itemize}

Il Proxy rimanda le richieste al RealSubject (se appropriato), in base al tipo di proxy.

\begin{figure}[H]
    \centering
    \includegraphics[width=0.75\linewidth]{assets/pattern/proxy/proxy-struttura.png}
    \caption{Class Diagram del pattern Proxy}
\end{figure}

\begin{figure}[H]
    \centering
    \includegraphics[width=0.75\linewidth]{assets/pattern/proxy/proxy-object-diagram.png}
    \caption{Object Diagram del pattern Proxy}
\end{figure}

\paragraph{Conseguenze} Il pattern Proxy consente quindi di:
\begin{itemize}
    \item Introdurre un livello di indirezione quando si accede ad un oggetto;
    \item Introdurre il \textbf{copy-on-write}: postporre il processo di copia fin quando l'oggetto non viene modificato.
\end{itemize}

\textbf{Esempio Java}
\begin{minted}[
    fontsize=\footnotesize,
    linenos,
]{java}
// Proxy implements Subject
public class ListaSicura<E> implements Lista<E> {

    private Lista<E> lista;
    private PermessiUtente pu; // RealSubject

    public ListaSicura(Lista<E> l, int nread, int nwrite) {
        lista = l;
        pu = new PermessiUtente(nread, nwrite);
    }

    @Override
    public void aggiungi(int index, E dato) throws IndexOutOfBoundsException {
        if(pu.getNumeroScritture() == 0) {
            throw new AccessoNonConsentitoException
        }
        pu.decrementaScritture();
        lista.aggiungi(index, dato);
    }

}
\end{minted}


\newpage

\section{Comportamentali}

I pattern comportamentali permettono di 
\begin{itemize}
    \item \textbf{Class Behavioral}: Utilizzare l'ereditarietà per distribuire il comportamento tra le classi.
    \item \textbf{Object Behavioral}: Utilizzare la composizione degli oggetti piuttosto che l'ereditarietà (per distribuire il comportamento tra le classi).
\end{itemize}

\subsection{Chain of Responsability}
\label{chain-of-responsability}

\textbf{Scopo}:  \\
\textbf{Raggio d'azione}: 

\paragraph{Definizione}

\paragraph{Problema}

\paragraph{Soluzione} 

\paragraph{Struttura e Conseguenze} 

\newpage
\subsection{Command (\textit{o Action, Transaction})}
\label{command}

\textbf{Scopo}: Comportamentale \\
\textbf{Raggio d'azione}: Oggetti

\paragraph{Definizione} Il pattern Command incapsula una richiesta in un oggetto, consentendo di parametrizzare i client con richieste diverse, accodare o mantenere uno storico delle richieste e gestire richieste cancellabili.

\paragraph{Motivazione} Talvolta è necessario inviare richieste ad oggetti senza conoscere nulla dell'operazione richiesta o del destinatario della richiesta. Per esempio, i toolkit per interfacce utente includono oggetti come pulsanti e menu che eseguono una richiesta in risposta all'input dell'utente, ma il toolkit non può implementare la richiesta esplicitamente nel pulsante o nel menu, perché solo le applicazioni che usano il toolkit sanno cosa dovrebbe essere fatto e su quale oggetto. Come progettisti di toolkit non abbiamo modo di conoscere il destinatario della richiesta o le operazioni che la eseguiranno.

\begin{multicols}{2}
    \begin{figure}[H]
        \centering
        \includegraphics[width=1\linewidth]{assets/pattern/command/command-esempio-1.png}
    \end{figure}
    \begin{figure}[H]
        \centering
        \includegraphics[width=1\linewidth]{assets/pattern/command/command-esempio-3.png}
    \end{figure}
    \columnbreak
    \begin{figure}[H]
        \centering
        \includegraphics[width=1\linewidth]{assets/pattern/command/command-esempio-2.png}
    \end{figure}
    \begin{figure}[H]
        \centering
        \includegraphics[width=1\linewidth]{assets/pattern/command/command-esempio-4.png}
    \end{figure}
\end{multicols}

Il pattern Command permette agli oggetti del toolkit di fare richieste ad oggetti applicativi non specificati trasformando la richiesta stessa in un oggetto che può essere memorizzato e passato come altri oggetti. La chiave di questo pattern è una classe astratta Command che dichiara un'interfaccia per eseguire operazioni, nella forma più semplice includendo un'operazione astratta Execute. Le sottoclassi concrete Command specificano una coppia destinatario-azione memorizzando il destinatario come variabile di istanza e implementando Execute per invocare la richiesta, dove il destinatario ha la conoscenza necessaria per eseguire la richiesta. I menu possono essere implementati facilmente con oggetti Command: ogni scelta in un menu è un'istanza della classe MenuItem, e quando l'utente seleziona un MenuItem, questo chiama Execute sul suo comando. I MenuItem non sanno quale sottoclasse di Command usano, mentre le sottoclassi Command memorizzano il destinatario della richiesta e invocano una o più operazioni su di esso. Notate come in tutti questi esempi il pattern Command disaccoppi l'oggetto che invoca l'operazione da quello che ha la conoscenza per eseguirla, dandoci grande flessibilità nella progettazione dell'interfaccia utente e permettendoci di sostituire comandi dinamicamente, supportare scripting di comandi componendo comandi in altri più grandi, tutto questo perché l'oggetto che emette una richiesta deve solo sapere come emetterla, non come verrà eseguita.

\paragraph{Applicabilità} È consigliabile utilizzare il pattern Command quando:
\begin{itemize}
    \item Si vogliono parametrizzare oggetti in base ad un'azione da eseguire;
    \item Si vogliono specificare, mettere in coda ed eseguire richieste in momenti diversi;
    \item Si vuole supportarte l'\textit{undo} delle operazioni;
    \item Si vogliono memorizzare le modifiche in modo che possano essere riapplicate in caso di crash del sistema;
    \item Si vuole strutturare un sistema attorno ad operazioni di alto livello basate su operazioni primitive,
\end{itemize}

\begin{figure}[H]
    \centering
    \includegraphics[width=0.75\linewidth]{assets/pattern/command/command-struttura.png}
    \caption{Class Diagram del pattern Command}
\end{figure}

\paragraph{Struttura} Il pattern Command è composto da:
\begin{itemize}
    \item \textbf{Command}: dichiara un’interfaccia per l’esecuzione di un’operazione generica.
    \item \textbf{ConcreteCommand} (PasteCommand, OpenCommand): definisce un legame fra un oggetto destinatario e un’azione. Implementa il metodo execute() invocando il metodo (i metodi) corrispondente sul Receiver.
    \item \textbf{Client} (Application): crea un’istanza concreta di Command e ne imposta il Receiver.
    \item \textbf{Invoker} (MenuItem): chiede a Command di portare a termine la richiesta
    \item \textbf{Receiver} (Document, Application): conosce il modo di svolgere le operazioni associate a una richiesta. Qualsiasi classe può essere vista come Receiver.
\end{itemize}

Il Client crea un oggetto ConcreteCommand e specifica chi lo riceve. L'Invoker salva il ConcreteCommand e effettua una richiesta chiamando la \textit{Execute()} sul Command (se l'\textit{undo()} è supportato il ConcreteCommand salva lo stato). Il ConcreteCommand invoca l'operazione e il suo ricettore la esegue.

\begin{figure}[H]
    \centering
    \includegraphics[width=0.75\linewidth]{assets/pattern/command/command-sequence.png}
    \caption{Sequence Diagram del pattern Command}
\end{figure}

\paragraph{Conseguenze} Command disaccoppia l’oggetto che invoca un’operazione da quello che conosce come portarla a termine.
Gli oggetti Command sono oggetti a tutti gli effetti, possono essere manipolati ed estesi con un qualsiasi altro oggetto.
È possibile comporre più comandi in un comando composito.
In generale, i comandi compositi sono un’istanza del pattern Composite.
Risulta facile aggiungere nuovi comandi poiché non è necessario modificare le classi esistenti


\newpage
\subsection{Data Transfer Object}
\label{data-transfer-object}

\textbf{Scopo}: Comportamentale \\
\textbf{Raggio d'azione}: Oggetti

\paragraph{Definizione} Oggetti che trasportano dati tra processi per ridurre il numero di chiamate ai metodi

\paragraph{Applicabilità} È consigliabile utilizzare il pattern Data Transfer Object quando:
\begin{itemize}
    \item Si vogliono ridurre i viaggi di andata e ritorno al server raggruppando più parametri in un’unica chiamata;
\end{itemize}

\begin{figure}[H]
    \centering
    \includegraphics[width=0.75\linewidth]{assets/pattern/dto/dto-struttura.png}
    \caption{Class Diagram del pattern DTO}
\end{figure}

\paragraph{Struttura} I partecipanti del pattern sono:
\begin{itemize}
    \item \textbf{DomainObject}: Oggetto del dominio
    \item \textbf{DataTransferObject}: Oggetto da trasferire
    \item \textbf{Assembler}: Componente che realizza il mapping tra DO e DTO
\end{itemize}

\newpage
\subsection{Interpreter}
\label{interpreter}

\textbf{Scopo}: Comportamentale \\
\textbf{Raggio d'azione}: Oggetti

\paragraph{Definizione} Il pattern Interpreter è un modello di progettazione che fornisce un modo per interpretare e valutare espressioni di un linguaggio, rappresentando la grammatica tramite una gerarchia di classi. Ogni classe corrisponde a una regola della grammatica e consente di comporre espressioni complesse in modo gerarchico.

Si definiscono classi di espressioni terminali (elementi di base) e non terminali (combinazioni di elementi), organizzate in una struttura ad albero simile al pattern Composite.

\begin{figure}[H]
    \centering
    \includegraphics[width=1\linewidth]{assets/pattern/interpreter/interpreter-struttura.png}
    \caption{Class Diagram del pattern Interpreter}
\end{figure}

\paragraph{Struttura e Conseguenze} Il pattern comprende:
\begin{itemize}
    \item \textbf{AbstractExpression}: Interfaccia o classe astratta che dichiara il metodo \texttt{interpreta}, comune a tutte le espressioni.
    \item \textbf{TerminalExpression}: Classi concrete che rappresentano i simboli di base della grammatica (foglie dell'albero), come numeri o variabili.
    \item \textbf{NonterminalExpression}: Classi concrete che rappresentano combinazioni di espressioni e gestiscono la logica di interpretazione delle sottoespressioni.
\end{itemize}

Le espressioni non terminali coordinano l'interpretazione delle sottoespressioni, aggregando i risultati per ottenere il valore finale dell'espressione.

\textbf{Vantaggi}:
\begin{itemize}
    \item Semplicità di implementazione: facilita la definizione e interpretazione di linguaggi semplici, rendendo il codice leggibile e manutenibile.
    \item Estensibilità: aggiungere nuove regole grammaticali è semplice, basta creare nuove classi.
\end{itemize}

\textbf{Svantaggi}:
\begin{itemize}
    \item Complessità: per linguaggi complessi, l'albero sintattico può diventare grande e difficile da gestire.
    \item Performance: la creazione e interpretazione degli alberi può essere inefficiente per linguaggi articolati.
\end{itemize}

\newpage

\subsection{Iterator (\textit{o Cursor})}
\label{iterator}

\textbf{Scopo}: Comportamentale \\
\textbf{Raggio d'azione}: Oggetti

\paragraph{Definizione} Il pattern fornisce un modo per accedere agli elementi di un oggetto aggregato in modo sequenziale senza esporne la rappresentazione sottostante.

\paragraph{Motivazione} Un oggetto aggregato come una lista dovrebbe fornire un modo per accedere ai suoi elementi senza esporre la sua struttura interna, e inoltre potresti voler attraversare la lista in modi diversi a seconda di ciò che vuoi realizzare, ma probabilmente non vuoi appesantire l'interfaccia della List con operazioni per attraversamenti diversi, anche se potessi anticipare quelli di cui avrai bisogno, e potresti anche aver bisogno di avere più di un attraversamento in corso sulla stessa lista.

\begin{figure}[H]
    \centering
    \includegraphics[width=0.5\linewidth]{assets/pattern/iterator/iterator-esempio-1.png}
\end{figure}

Il pattern Iterator ti permette di fare tutto questo prendendo la responsabilità dell'accesso e dell'attraversamento dall'oggetto lista e mettendola in un oggetto iteratore, dove la classe Iterator definisce un'interfaccia per accedere agli elementi della lista e un oggetto iteratore è responsabile di tenere traccia dell'elemento corrente, sapendo quali elementi sono già stati attraversati. Separare il meccanismo di attraversamento dall'oggetto List ci permette di definire iteratori per diverse politiche di attraversamento senza enumerarle nell'interfaccia List, ma notate che l'iteratore e la lista sono accoppiati e il cliente deve sapere che è una lista che viene attraversata invece di qualche altra struttura aggregata, quindi il cliente si impegna con una particolare struttura aggregata.

\begin{figure}[H]
    \centering
    \includegraphics[width=0.75\linewidth]{assets/pattern/iterator/iterator-esempio-2.png}
\end{figure}

Sarebbe meglio se potessimo cambiare la classe aggregata senza cambiare il codice cliente, e possiamo farlo generalizzando il concetto di iteratore per supportare iterazione polimorfa definendo una classe AbstractList che fornisce un'interfaccia comune per manipolare liste e una classe Iterator astratta che definisce un'interfaccia di iterazione comune, permettendo poi di definire sottoclassi Iterator concrete per diverse implementazioni di lista, rendendo così il meccanismo di iterazione indipendente dalle classi aggregate concrete e usando un metodo factory come CreateIterator per connettere le due gerarchie di classi.

\paragraph{Applicabilità} È consigliabile utilizzare il pattern Iterator quando:
\begin{itemize}
    \item Si vuole accedere al contenuto di un oggetto aggregato senza esporne la rappresentazione interna;
    \item Si vogliono supportare più modi di attraversare oggetti aggregati.
    \item Si vuole fornire un'interfaccia uniforme per l'attraversamento di diverse strutture aggregate (ovvero, per supportare l'\textbf{iterazione polimorfica}).
\end{itemize}

\begin{figure}[H]
    \centering
    \includegraphics[width=0.75\linewidth]{assets/pattern/iterator/iterator-struttura.png}
    \caption{Class Diagram del pattern Iterator}
\end{figure}

\paragraph{Struttura} Il pattern Iterator è composto da:
\begin{itemize}
    \item \textbf{Iterator}: definisce un’interfaccia per attraversare l’insieme degli elementi di un contenitore e accedere ai singoli elementi.
    \item \textbf{ConcreteIterator}: implementa l’interfaccia Iterator tenendo traccia della posizione corrente nel contenitore e calcolando qual `e l’elemento successivo nella sequenza di attraversamento.
    \item \textbf{Aggregate}: definisce un’interfaccia per creare un oggetto Iterator.
    \item \textbf{ConcreteAggregate}: implementa l’interfaccia di creazione dell’Iterator e ritorna un’istanza appropriata di ConcreteIterator.
\end{itemize}

\paragraph{Conseguenze} Il pattern Iterator consente quindi di:
\begin{itemize}
    \item Semplificare l'interfaccia Aggregate: estraendo algoritmi di attraversamento voluminosi in classi separate.
    \item Supportare variazioni nell'attraversamento di un aggregato;
\end{itemize}

È bene notare che applicare il modello può risultare eccessivo se l'applicazione utilizza solo collezioni semplici (es. Liste, Set).

\begin{figure}[H]
    \centering
    \includegraphics[width=0.75\linewidth]{assets/pattern/iterator/iterator-sequence.drawio.png}
    \caption{Sequence Diagram del pattern Iterator}
\end{figure}

ConcreteIterator posside un'istanza della collezione (passata a costruttore).

\newpage
\subsection{Mediator}
\label{mediator}

\textbf{Scopo}: Comportamentale \\
\textbf{Raggio d'azione}: Oggetti

\paragraph{Definizione} Il pattern Mediator definisce un oggetto che incapsula il modo in cui un insieme di oggetti interagisce. Il mediatore promuove l’accoppiamento libero impedendo agli oggetti di riferirsi esplicitamente tra loro e consente di variare la loro interazione in modo indipendente.

\paragraph{Nota} È bene notare che la POO incoraggia la distribuzione di comportamento tra gli oggetti, portando ad una struttura con molte connesioni tra oggetti. Le molte interconnessioni rendono meno probabile che un oggetto possa funzionare senza il supporto di altri.

\begin{figure}[H]
    \centering
    \includegraphics[width=0.8\linewidth]{assets/pattern/mediator/mediator-struttura.png}
    \caption{Class Diagram del pattern Mediator}
\end{figure}

\paragraph{Struttura e Conseguenze} 
\begin{itemize}
    \item \textbf{Mediator}: definisce un’interfaccia che espone i metodi che abilitano la comunicazione tra i componenti in relazione o agisce come punto centrale per la gestione delle interazioni tra i vari partecipanti. Il Mediatore conosce e mantiene i riferimenti agli oggetti dei partecipanti e facilita la loro comunicazione. Gestisce le dipendenze e incapsula il comportamento.
    \item \textbf{ConcreteMediator}: implementa l’interfaccia Mediator e, nel definire i metodi dell’interfaccia, realizza uno specifico comportamento di cooperazione tra i componenti. Gestisce effettivamente la comunicazione tra i partecipanti, implementando i metodi definiti nell'interfaccia del Mediatore. Essa conosce i partecipanti e regola il flusso delle comunicazioni tra di essi.
    \item \textbf{Colleague}: definisce l’interfaccia che rappresenta un componente del sistema che deve comunicare tramite il Mediatore. Ogni Colleague conosce solo il Mediatore e non ha una conoscenza diretta degli altri Colleague. Essi inviano e ricevono messaggi tramite il Mediatore.
    \item \textbf{ConcreteColleague}: classe concreta che implementa l'interfaccia dei Colleague. Ogni ConcreteColleague è un componente specifico del sistema che deve comunicare con gli altri Colleague tramite il Mediatore.
\end{itemize}

Il pattern mediator permette di limitare la sottoclasse del comportamento, separare i colleghi, semplificare i protocolli degli oggetti, facilitare l’aggiunta di nuovi componenti, astrarre la cooperazione tra oggetti, favorire il riutilizzo dei Mediator e centralizzare il controllo delle interazioni.

\begin{figure}[H]
    \centering
    \includegraphics[width=0.8\linewidth]{assets/pattern/mediator/mediator-sequence.drawio.png}
    \caption{Sequence Diagram del pattern Mediator}
\end{figure}

\paragraph{Pattern correlati}
\begin{itemize}
    \item \textbf{Observer} (\ref{observer})
    \item \textbf{Facade} (\ref{facade})
    \item \textbf{Command} (\ref{command})
    \item \textbf{Chain of Responsibility} (\ref{chain-of-responsability})
    \item \textbf{Strategy} (\ref{strategy})
\end{itemize}

\newpage
\subsection{Memento (\textit{o Token})}
\label{memento}

\textbf{Scopo}: Comportamentale \\
\textbf{Raggio d'azione}: Oggetti

\paragraph{Definizione} Il pattern, senza violare l'incapsulamento, cattura ed esternalizza lo stato interno di un oggetto in modo che l'oggetto possa essere ripristinato a questo stato in un secondo momento.

\paragraph{Motivazione} A volte è necessario registrare lo stato interno di un oggetto, cosa richiesta quando si implementano checkpoint e meccanismi di undo che permettono agli utenti di annullare operazioni tentative o recuperare da errori. Devi salvare informazioni di stato da qualche parte per poter ripristinare gli oggetti ai loro stati precedenti, ma gli oggetti normalmente incapsulano parte o tutto il loro stato, rendendolo inaccessibile ad altri oggetti e impossibile da salvare esternamente, e esporre questo stato violerebbe l'incapsulamento, compromettendo l'affidabilità e l'estensibilità dell'applicazione. Considera per esempio un editor grafico che supporta la connettività tra oggetti dove un utente può connettere due rettangoli con una linea e i rettangoli rimangono connessi quando l'utente muove uno di essi, con l'editor che assicura che la linea si estenda per mantenere la connessione. 

\begin{multicols}{2}
    \begin{figure}[H]
        \centering
        \includegraphics[width=1\linewidth]{assets/pattern/memento/memento-esempio-1.png}
    \end{figure}
    \columnbreak
    \vline
    \begin{figure}[H]
        \centering
        \includegraphics[width=1\linewidth]{assets/pattern/memento/memento-esempio-2.png}
    \end{figure}
\end{multicols}

Un modo noto per mantenere relazioni di connettività tra oggetti è con un sistema di risoluzione di vincoli incapsulato in un oggetto ConstraintSolver che registra le connessioni mentre vengono fatte e genera equazioni matematiche che le descrivono. Supportare l'undo in questa applicazione non è facile come può sembrare: un modo ovvio per annullare un'operazione di movimento è memorizzare la distanza originale spostata e muovere l'oggetto indietro di una distanza equivalente, tuttavia questo non garantisce che tutti gli oggetti appariranno dove erano prima. In generale, l'interfaccia pubblica del ConstraintSolver potrebbe essere insufficiente per permettere l'inversione precisa dei suoi effetti su altri oggetti, e il meccanismo di undo deve lavorare più strettamente con ConstraintSolver per ristabilire lo stato precedente, ma dovremmo anche evitare di esporre gli interni del ConstraintSolver al meccanismo di undo. Possiamo risolvere questo problema con il pattern Memento: un memento è un oggetto che memorizza un'istantanea dello stato interno di un altro oggetto—l'originatore del memento—dove il meccanismo di undo richiederà un memento dall'originatore quando deve fare un checkpoint dello stato dell'originatore, e l'originatore inizializza il memento con informazioni che caratterizzano il suo stato corrente, dove solo l'originatore può memorizzare e recuperare informazioni dal memento poiché il memento è "opaco" ad altri oggetti.

\newpage

\paragraph{Applicabilità} È opportuno usare il pattern Memento quando:
\begin{itemize}
    \item Si deve memorizzare un'istantanea (totale o parziale) dello stato di un oggetto, così da poterla ripristinare in un secondo tempo;
    \item Si vogliono proteggere dettagli implementativi che violerebbero l'incapsulamento.
\end{itemize}

\begin{figure}[H]
    \centering
    \includegraphics[width=1\linewidth]{assets/pattern/memento/memento-struttura.png}
    \caption{Class Diagram del pattern Memento}
\end{figure}

\paragraph{Struttura} Il pattern è composto da
\begin{itemize}
    \item \textbf{Memento}: memorizza lo stato interno dell’oggetto Originator, non permette l’accesso alla sua struttura dati se non all’Originator, presentando così due interfacce.
    \item \textbf{Originator} Crea un Memento contenente un’istantanea del proprio stato interno corrente. Usa un Memento per ripristinare il proprio stato interno tramite l’interfaccia estesa di Memento, con la quale è possibile accedere a tutti i dati necessari. Idealmente solo l’Originator che ha prodotto il Memento ha il permesso di accedere allo stato interno del Memento. 
    \item \textbf{Caretaker} È responsabile di memorizzare i Memento. Non invoca operazioni né esamina i contenuti di un Memento, vede un’interfaccia ridotta di Memento e può solo passare il Memento ad altri oggetti.
\end{itemize}

In altre implementazioni, la classe Memento è annidata all'interno di Originator. Ciò consente all'Originator di accedere ai campi e ai metodi del Memento, anche se sono dichiarati privati. D'altra parte, il Caretaker ha un accesso molto limitato ai campi e ai metodi del ricordo, il che gli consente di archiviare i ricordi in uno Stack ma di non manometterne lo stato.

\paragraph{Conseguenze} Il pattern Mediator consente quindi di:
\begin{itemize}
    \item Preservare i confini dell'incapsulamento
    \item Semplificare l'Originator
    \item Definire interfacce \textit{narrow and wide}
\end{itemize}

È bene notare che l'uso dei Memento potrebbe essere costoso, esistono dei costi nascosti (soprattutto in termini di memoria, ma anche di gestione).

\begin{figure}[H]
    \centering
    \includegraphics[width=0.75\linewidth]{assets/pattern/memento/memento-sequence.png}
    \caption{Sequence Diagram del pattern Memento}
\end{figure}

\newpage
\subsection{Observer (\textit{o Dependents, Publish-Subscribe})}
\label{observer}

\textbf{Scopo}: Comportamentale \\
\textbf{Raggio d'azione}: Oggetti

\paragraph{Definizione} Il pattern Observer permette di definire una dipendenza uno a molti tra oggetti, in modo tale che se un oggetto cambia il suo stato, tutti gli oggetti dipendenti da questo siano notificati e aggiornati automaticamente.

\begin{figure}[H]
    \centering
    \includegraphics[width=1\linewidth]{assets/pattern/observer/observer-struttura.png}
\end{figure}

\paragraph{Struttura} Il pattern è composto da:
\begin{itemize}
    \item \textbf{Subject}: conosce i propri osservatori; un numero qualunque di oggetti Observer può osservare un soggetto. Fornisce un’interfaccia per registrare e cancellare le registrazioni degli oggetti Observer. 
    \item \textbf{Observer}: fornisce un’interfaccia di notifica per gli oggetti a cui devono essere notificati i cambiamenti nel Subject. 
    \item \textbf{ConcreteSubject}: contiene lo stato a cui gli oggetti 
     ConcreteObserver sono interessati. Inoltra una notifica ai suoi Observer quando il proprio stato si modifica. 
    \item \textbf{ConcreteObserver}: memorizza un riferimento a un oggetto ConcreteSubject oppure ottiene dinamicamente il riferimento all’oggetto Subject da cui ha origine la notifica in caso in esso cui osservi più Subject. Contiene informazioni che devono essere sincronizzate con lo/gli stato/i del/i Subject. Implementa l’interfaccia Observer per ricevere le notifiche del Subject.
\end{itemize}

\begin{figure}[H]
    \centering
    \includegraphics[width=1\linewidth]{assets/pattern/observer/observer-activity.png}
\end{figure}

\paragraph{Interazioni} ConcreteSubject notifica i propri Observer quando avviene un cambiamento che potrebbe rendere il loro stato non inconsistente rispetto al proprio. Dopo essere stato informato di un cambiamento nel ConcreteSubject, un osservatore concreto può chiedere al Subject informazioni sul suo stato. ConcreteObserver usa questa informazione per riconciliare il suo stato con quello del subject.

\newpage
\subsection{State (\textit{o Object for States})}
\label{state}

\textbf{Scopo}: Comportamentale \\
\textbf{Raggio d'azione}: Oggetti

\paragraph{Definizione} Il pattern State consente ad un oggetto di modificare il proprio comportamento quando cambia il suo stato interno. L'oggetto \textit{sembrerà} cambiare classe.

\paragraph{Motivazione} Considera una classe TCPConnection che rappresenta una connessione di rete, dove un oggetto TCPConnection può essere in uno di diversi stati: Established, Listening, Closed. Quando un oggetto TCPConnection riceve richieste da altri oggetti, risponde diversamente a seconda del suo stato corrente: per esempio, l'effetto di una richiesta Open dipende dal fatto che la connessione sia nel suo stato Closed o nel suo stato Established. 

\begin{figure}[H]
    \centering
    \includegraphics[width=0.75\linewidth]{assets/pattern/state/state-esempio.png}
\end{figure}

Il pattern State descrive come TCPConnection possa esibire comportamenti diversi in ogni stato, e l'idea chiave in questo pattern è introdurre una classe astratta chiamata TCPState per rappresentare gli stati della connessione di rete, dove la classe TCPState dichiara un'interfaccia comune a tutte le classi che rappresentano stati operazionali diversi, e le sottoclassi di TCPState implementano comportamento specifico dello stato, per esempio le classi TCPEstablished e TCPClosed implementano comportamento particolare agli stati Established e Closed di TCPConnection. La classe TCPConnection mantiene un oggetto stato (un'istanza di una sottoclasse di TCPState) che rappresenta lo stato corrente della connessione TCP, e TCPConnection delega tutte le richieste specifiche dello stato a questo oggetto stato, usando la sua istanza di sottoclasse TCPState per eseguire operazioni particolari allo stato della connessione. Ogni volta che la connessione cambia stato, l'oggetto TCPConnection cambia l'oggetto stato che usa: quando la connessione va da established a closed, per esempio, TCPConnection sostituirà la sua istanza TCPEstablished con un'istanza TCPClosed.

\paragraph{Applicabilità} È opportuno usare il pattern State quando:
\begin{itemize}
    \item Il comportamento di un oggetto dipende dal suo stato e deve cambiare in fase di esecuzione a seconda di esso.
    \item Le operazioni hanno istruzioni condizionali grandi e multiparte che dipendono dallo stato dell'oggetto: questo stato è solitamente rappresentato da una o più costanti enumerate.
    \item Diverse operazioni contengono la stessa struttura condizionale: il modello State inserisce ogni ramo condizionale in una classe separata (per evitare di applicazioni azioni diverse con comandi quali switch). Ciò consente di trattare lo stato dell'oggetto come un oggetto a sé stante che può variare indipendentemente dagli altri oggetti.
\end{itemize}

\begin{figure}[H]
    \centering
    \includegraphics[width=0.75\linewidth]{assets/pattern/state/state-struttura.png}
    \caption{Class Diagram del pattern}
\end{figure}

\paragraph{Struttura} Il pattern è composto da:
\begin{itemize}
    \item \textbf{Context}: definisce l’interfaccia utilizzata dai client. Mantiene un riferimento ad un’istanza di una classe che implementa l’interfaccia State e che rappresenta lo stato corrente.
    \item \textbf{State}: Definisce un’interfaccia che incapsula il comportamento associato ad uno stato particolare di Context.
    \item \textbf{ConcreteState}: Ogni classe che occupa questo ruolo definisce un particolare comportamento associato ad uno stato di Context.
\end{itemize}

\paragraph{Conseguenze} Il pattern State consente quindi di:
\begin{itemize}
    \item Localizzare il comportamento specifico dello stato e suddividere il comportamento per stati diversi.
    \item Rendere esplicite le transizioni di stato (atomiche dal p.d.v. di Context)
    \item Condividere gli oggetti State
\end{itemize}

È bene notare che l'applicazione del modello risulta eccessiva per applicazioni che presentano pochi stati o che cambiano raramente.

\paragraph{Pattern correlati} Può essere correlato ai pattern Flyweight (\ref{flyweight}, utilizzato per la condivisione di oggetti State) e Singleton (\ref{singleton}, spesso gli oggetti State sono dei Singleton)

\newpage
\subsection{Strategy}
\label{strategy}

\textbf{Scopo}: Comportamentale \\
\textbf{Raggio d'azione}: Oggetti

\paragraph{Definizione} Il pattern Strategy, a differenza di Template (\ref{template}), in cui si congela l'algoritmo per risolvere il problema a seconda del tipo di specifica, cambia l'algoritmo a seconda del tipo concreto. Ha lo scopo di definire una famiglia di algoritmi, incapsularli e renderli interscambiabili.

\begin{figure}[H]
    \centering
    \includegraphics[width=0.5\linewidth]{assets/pattern/strategy/strategy-struttura.png}
    \caption{Class Diagram del pattern Strategy}
\end{figure}

\paragraph{Struttura e Conseguenze} Il pattern è composto da:
\begin{itemize}
    \item \textbf{Context}: mantiene un riferimento a una delle ConcreteStrategy e comunica con questo oggetto solo tramite l’interfaccia Strategy.
    \item \textbf{Strategy}: interfaccia comune a tutte le ConcreteStrategy. Dichiara un metodo che il contesto utilizza per eseguire una strategia.
    \item \textbf{ConcreteStrategy}: implementano diverse varianti di un algoritmo utilizzato dal Context.
    \item \textbf{Client}: crea un oggetto ConcreteStrategy e lo passa al Context, il quale espone un setter che consente ai Client di sostituire la Strategy associata al Context a runtime.
\end{itemize}

\paragraph{Applicabilità} Il pattern Strategy è utile quando si desidera utilizzare diverse varianti di un algoritmo all'interno di un oggetto ed essere in grado di passare da un algoritmo all'altro a runtime.

Risulta efficace quando si hanno molte classi simili che differiscono solo nel modo in cui eseguono alcuni comportamenti (queste classi spesso contengono istruzioni condizionali massicce per passare da un'algoritmo all'altro).

In più permette di isolare la logica di business di una classe dai dettagli di implementazione degli algoritmi che potrebbero non essere così importanti nel contesto di tale logica.

Può essere considerato come un'estensione del pattern State (\ref{state}), entrambi sono basati sulla \textbf{composizione}: modificano il comportamento del contesto delegando parte del lavoro agli oggetti helper. Sebbene, il pattern State non limiti le dipendenze tra stati concreti, consentendo loro di alterare a piacimento lo stato del contesto.

\begin{figure}[H]
    \centering
    \includegraphics[width=0.7\linewidth]{assets/pattern/strategy/strategy-sequence.drawio.png}
    \caption{Sequence Diagram del pattern Strategy}
\end{figure}

\paragraph{Vantaggi}
\begin{itemize}
    \item È possibile scambiare gli algoritmi utilizzati all'interno di un oggetto in fase di esecuzione.
    \item È possibile isolare i dettagli di implementazione di un algoritmo dal codice che lo utilizza.
    \item Si possono introdurre nuove strategie senza dover modificare il contesto.
\end{itemize}

\paragraph{Svantaggi}
\begin{itemize}
    \item I client devono essere consapevoli delle differenze tra le strategie per poter selezionare quella più adeguata.
    \item Se si hanno solo un paio di algoritmi che cambiano raramente, l'uso del pattern può introdurre complessità non necessaria con nuove classi e interfacce.
\end{itemize}


\newpage
\subsection{Template Method}
\label{template-method}

\textbf{Scopo}: Comportamentale \\
\textbf{Raggio d'azione}: Classi

\paragraph{Definizione} Il pattern definisce la struttura di un algoritmo all'interno di un metodo, delegando alcuni passi alle sottoclassi (motivo per cui è un pattern class-based).

Consente, introducendo sottoclassi diverse, di cambaire il modo in cui i sottopassi vengono eseguiti, lasciando inalterata la struttura dell'algoritmo che è stato inglobato.

\paragraph{Motivazione} Considera un framework applicativo che fornisce classi Application e Document, dove la classe Application è responsabile di aprire documenti esistenti memorizzati in un formato esterno come un file, e un oggetto Document rappresenta le informazioni in un documento una volta che è stato letto dal file. Le applicazioni costruite con il framework possono creare sottoclassi di Application e Document per soddisfare bisogni specifici: per esempio, un'applicazione di disegno definisce sottoclassi DrawApplication e DrawDocument, mentre un'applicazione foglio di calcolo definisce sottoclassi SpreadsheetApplication e SpreadsheetDocument. La classe astratta Application definisce l'algoritmo per aprire e leggere un documento nella sua operazione OpenDocument che definisce ogni passo per aprire un documento: controlla se il documento può essere aperto, crea l'oggetto Document specifico dell'applicazione, lo aggiunge al suo insieme di documenti, e legge il Document da un file. Chiamiamo OpenDocument un template method, che definisce un algoritmo in termini di operazioni astratte che le sottoclassi sovrascrivono per fornire comportamento concreto, dove le sottoclassi Application definiscono i passi dell'algoritmo che controllano se il documento può essere aperto (CanOpenDocument) e che creano il Document (DoCreateDocument), mentre le classi Document definiscono il passo che legge il documento (DoRead).

\begin{figure}[H]
    \centering
    \includegraphics[width=0.75\linewidth]{assets/pattern/template-method/template-esempio.png}
\end{figure}

Il template method definisce anche un'operazione che permette alle sottoclassi Application di sapere quando il documento sta per essere aperto (AboutToOpenDocument), nel caso sia di loro interesse. Definendo alcuni dei passi di un algoritmo usando operazioni astratte, il template method fissa il loro ordine, ma permette alle sottoclassi Application e Document di variare quei passi per soddisfare i loro bisogni.

\paragraph{Applicabilità} Il pattern Template Method è utile quando:
\begin{itemize}
    \item Si desidera implementare una volta sola le parti invarianti di un algoritmo e lasciare alle sottoclassi l'implementazione dei comportamenti che possono variare.
    \item I comportamenti comuni tra le sottoclassi devono essere fattorizzati e localizzati in una classe comune per evitare la duplicazione del codice;
    \item Si vogliono controllare le estensioni delle sottoclassi;
\end{itemize}

\begin{figure}[H]
    \centering
    \includegraphics[width=0.5\linewidth]{assets/pattern/template-method/template-struttura.png}
    \caption{Class Diagram del pattern Template}
\end{figure}

\paragraph{Struttura} Il pattern è composto da:
\begin{itemize}
    \item \textbf{AbstractClass} (Applicazione): definisce operazioni primitive astratte che le sottoclassi concrete definiscono per implementare le fasi di un algoritmo.Implementa un metodo modello che definisce lo scheletro di un algoritmo. Il metodo modello richiama operazioni primitive, operazioni definite in AbstractClass o quelle di altri oggetti.
    \item \textbf{ConcreteClass} (MyApplication): implementa le operazioni primitive per eseguire le fasi dell'algoritmo specifiche della sottoclasse.
\end{itemize}

ConcreteClass si basa su AbstractClass per implementare i passaggi invarianti dell'algoritmo.

\begin{figure}[H]
    \centering
    \includegraphics[width=0.5\linewidth]{assets/pattern/template-method/template-sequence.drawio.png}
    \caption{Sequence Diagram del pattern Template}
\end{figure}

\paragraph{Conseguenze} Il pattern Template Method consente quindi di:
\begin{itemize}
    \item Riutilizzare codice già scritto;
    \item Applicare il \textbf{principio di Hollywood} (Don't call us, we'll call you): la classe padre chiama le operazioni di una sottoclasse e non viceversa;
\end{itemize}

Il pattern permette di rimuovere codice duplicato inserendolo nella AbstractClass, però alcuni client potrebbero essere limitati dallo scheletro fornito di un dato algoritmo. In più i metodi tendono ad essere più difficili da mantenere all'aumentare dei passaggi forniti.



\newpage
\subsection{Visitor}
\label{visitor}

\textbf{Scopo}: Comportamentale \\
\textbf{Raggio d'azione}: Oggetti

\paragraph{Definizione} Il pattern Visitor rappresenta un’operazione da eseguire sugli oggetti di una struttura e consente di definire nuove operazioni senza modificare le classi degli oggetti su cui opera (vantaggioso).

\paragraph{Vantaggi} Da la possibilità di aggiungere dinamicamente delle nuove operazioni, senza interrompere l'esecuzione perchè le operazioni vengono integrate in degli oggetti.

\begin{figure}[H]
    \centering
    \includegraphics[width=1\linewidth]{assets/pattern/visitor/visitor-struttura.png}
    \caption{Struttura del pattern}
\end{figure}

\newpage

\paragraph{Struttura e Interazioni} Il pattern è composto da:
\begin{itemize}
    \item \textbf{Visitor}: dichiara un metodo di visita per ogni classe concreta della struttura, il nome e la firma del metodo identificano la classe che invia la richiesta di visita al visitor in modo che il visitor possa identificare la classe dell’elemento che sta per visitare 
    \item \textbf{ConcreteVisitor}: implementa i metodi definiti da Visitor. Ogni metodo implementa un frammento dell’algoritmo definito per la struttura. In più fornisce il contesto per l’algoritmo e memorizza il suo stato locale che accumula durante la visita della struttura. 
    \item \textbf{Element}: definisce il metodo \textit{accept()} che riceve un Visitor.
    \item \textbf{ConcreteElements}: implementano il metodo \textit{accept()} che riceve un Visitor.
    \item \textbf{ObjectStructure}: può enumerare gli elementi che la costituiscono. può essere realizzata come un Composite o come una collezione di elementi.
\end{itemize}

\begin{figure}[H]
    \centering
    \includegraphics[width=1\linewidth]{assets/pattern/visitor/visitor-activity.png}
    \caption{Activity Diagram del pattern Visitor}
\end{figure}

\paragraph{Correlazioni} Il pattern Visitor può essere utilizzato per applicare un'operazione su una struttura di oggetti implementata attraverso il pattern Composite (\ref{composite}) oppure per eseguire l'effettiva interpretazione di un'espressione (Interpreter, \ref{interpreter}).


\newpage