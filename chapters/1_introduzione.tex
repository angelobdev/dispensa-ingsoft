\chapter{Intro e concetti di base}

Gli obiettivi dell'ingegneria del software sono:
\begin{enumerate}
    \item Conoscere natura e qualità del software
    \item Saper rappresentare graficamente le fasi di sviluppo di progetto software
    \item Saper analizzare una soluzione progettuale (forza e debolezza)
    \item Sapersi muovere tra:
    \begin{itemize}
        \item Progettazione
        \item Implementazione
        \item Testing
        \item Manutenzione
    \end{itemize}
    \item Conoscenza delle metodologie di progettazione e sviluppo
\end{enumerate}

L'Ingegneria del Software si occupa di creare \textbf{sistemi software complessi} per cui è richiesto il lavoro di uno o più team.

Questi sistemi restano in \textit{vita} per diversi anni e subiscono modifiche allo scopo di:
\begin{enumerate}
    \item Eliminare difetti
    \item Migliorare caratteristiche
    \item Introdurre nuove funzionalità
    \item Rimuovere caratteristiche obsolete
\end{enumerate}

\begin{table}[h!]
    \centering
    \begin{tabularx}{\textwidth}{|X|X|}
        \hline
        \textbf{Programmazione} & \textbf{Ingegneria del Software} \\
        \hline
        Scrittura di un programma completo & Progettazione e sviluppo di componenti che dovranno interagire tra loro (scritti anche da altri) \\
        \hline
        Attività individuale & Attività di gruppo \\
        \hline
    \end{tabularx}
    \caption{Definizione di Parnas}
\end{table}

\newpage

L'\textit{ingegnere del software} deve:
\begin{itemize}
    \item Saper programmare in uno o più linguaggi
    \item Avere familiarità con più approcci di progettazione
    \item Saper tradurre richieste vaghe in specifiche precise
    \item Sapere interagire con l'utente
    \item Districarsi attraverso diversi livelli di astrazione progettuale
    \item Saper costruire modelli e ragionarci al fine di scegliere un buon compromesso (\textit{trade-off})
    \item Saper operare all'interno di un team
\end{itemize}

\section{Ciclo di vita del software}
Costituito da diverse fasi (aventi come output un artefatto da trasmettere alla fase successiva) atte a sviluppare, implementare e consegnare un prodotto più affine possibile alle richieste del cliente.

\paragraph{Ciclo a cascata (Modello waterfall)}

\begin{enumerate}
    \item \textbf{Analisi e specifica dei requisiti}
    \begin{itemize}
        \item Definizione di costi e benefici della realizzazione del software
        \item Identificazione dei requisiti del sistema (espressi in una notazione formale comprensibile dal cliente)
        \item Attori: Cliente, Analisti di mercato, team di sviluppo
    \end{itemize}
    \item \textbf{Progettazione e specifica di sistema}, suddivisa in:
    \begin{itemize}
        \item \textit{Progettazione architetturale}: organizzazione del sistema in termini di componenti (di alto livello) e loro interazioni
        \item \textit{Progettazione di dettaglio}: scomposizione delle componenti in modulo di basso livello accuratamente definito
        \item Output: documento di specifica
    \end{itemize}
    \item \textbf{Codifica e test di modulo}
    \begin{itemize}
        \item Implementazione del sistema funzionante (composto da componenti progettati nelle fasi precedenti)
        \item Produzione di codice di test e successiva fase di test dei moduli
    \end{itemize}
    \item \textbf{Integrazione e test di sistema}: Integrazione e test dei moduli precedentemente sviluppati
    \item \textbf{Consegna e manutenzione}: Consegna al cliente e successiva fase di \textit{manutenzione} (raccolta di modifiche apportate al sistema dopo la sua consegna)
\end{enumerate}

\section{Natura e Qualità del software}

\subsection{Natura del software}

Prodotto dell'ingegneria del software: \textit{sistema software}

Distinto poiché \textit{duttile}: può essere direttamente modificato senza riconsiderare il progetto (rischioso).

Il costo del software è determinato da:
\begin{itemize}
    \item Progettazione
    \item Implementazione
    \item Manutenzione
\end{itemize}

\subsection{Qualità del software}

\textbf{Esterne}: visibili agli utenti del sistema.
\textbf{Interne}: visibili agli sviluppatori, importanti per il conseguimento delle esterne.

\begin{center}
  $\ast$~$\ast$~$\ast$
\end{center}

\paragraph{Correttezza} Un sistema software è \textit{corretto} soddisfa i requisiti (funzionali e non funzionali), espressi in maniera non ambigua. Valutabile tramite: testing e/o approcci analitici.

\paragraph{Affidabilità} Un sistema software è \textit{affidabile} se opera come previsto. Definita come \textit{probabilità che il sistema si comporti come attesa per un certo intervallo di tempo}. Affidabilità implica correttezza (ma non il contrario).

\paragraph{Robustezza} Un sistema software è \textit{robusto} se si comporta "bene" anche in casi non previsti dalle specifiche.

\paragraph{Prestazioni} Il livello di \textit{prestazioni} varia in base a: efficienza, risorse utilizzate, altro. È valutato tramite: misura, analisi di complessità e simulazione. Nel caso del processo di sviluppo prende il nome di \textbf{produttività}.

\paragraph{Usabilità} Un software è detto \textit{usabile} se è \textbf{user-friendly}. Dipende dall'implementazione dell'interfaccia.

\paragraph{Verificabilità} Un sistema software è \textit{verificabile} se è possibile verificarne correttezza e prestazioni. Può essere qualità esterna per sistemi critici. Svolta tramite analisi formale e testing o moduli di monitoraggio. Dipende da progettazione e linguaggi di programmazione.

\paragraph{Manutenibilità} Un sistema software è \textit{manutenibile} se è possibile: eliminare difetti, migliorare il funzionamento, introdurre nuove funzioni. È la qualità più costosa. Divisa in:
\begin{itemize}
    \item \textit{Adattiva}: modifiche richieste dall'ambiente esterno
    \item \textit{Correttiva}: eliminazione di errori
    \item \textit{Perfettiva}: interventi migliorativi o nuove funzioni
\end{itemize}

\paragraph{Riparabilità} Un sistema software è \textit{riparabile} se è possibile correggere difetti con quantità ragionevoli di lavoro. Dipende dalla modularità del software.

\paragraph{Evolvibilità} Un sistema software è \textit{evolvibile} se è capace di sostenere modifiche evolutive in maniera controllata e ordinata. Dipende dalle fasi di progettazione. È una qualità che si degrada all'aumentare delle iterazioni (modifiche).

\paragraph{Riusabilità} Un sistema software \textit{riusabile} se è possibile utilizzarne (una o più) parti per dare origine ad un altro prodotto (es. librerie), è un obiettivo della POO e si applica anche alle fasi di progettazione.

\paragraph{Portabilità} Un sistema software è \textit{portabile} se è possibile eseguirlo in ambienti (hardware o S.O.) diversi. È favorita dall'uso di linguaggi interpretati (es. Python, Java).

\paragraph{Comprensibilità} Un sistema software è \textit{comprensibile} in base alla complessità di progettazione e implementazione. Qualità che aiuta evolvibilità e verificabilità.

\paragraph{Interoperabilità} Un sistema software è \textit{interoperabile} se è capace di coesistere e/o cooperare con altri sistemi. Si raggiunge tramite l'uso di interfacce standard (es. plugin browser). Importante nei sistemi open-source.

\paragraph{Produttività} Un sistema software è più (o meno) \textit{produttivo} in base ad efficienza e prestazioni del processo di produzione. Influenzata dalla riusabilità.

\paragraph{Tempestività} Un sistema software è \textit{tempestivo} se è consegnato nei tempi previsti. Non è sempre una qualità. Un sistema software più completo consegnato in ritardo è meglio di un sistema incompleto consegnato in tempo. La consegna incrementale è una buona soluzione.

\paragraph{Visibilità} Un processo di sviluppo è \textit{visibile} se tutti i passi e lo stato corrente sono correttamente documentati. È fondamentale per valutare: azioni progettuali, variazioni del personale.

\newpage